\documentclass{article}
\usepackage{amsmath}
\title{Face Tracker}
\author{wandyezj}

\usepackage{comment}
\usepackage[yyyymmdd]{datetime}
\renewcommand{\dateseparator}{--}


\usepackage{verbatim}

\usepackage[margin=0.5in]{geometry}

\usepackage{enumitem}

\usepackage{listings}
\usepackage{color}

\definecolor{dkgreen}{rgb}{0,0.6,0}
\definecolor{gray}{rgb}{0.5,0.5,0.5}
\definecolor{mauve}{rgb}{0.58,0,0.82}

	\lstset{frame=tb,
	language=Java,
	aboveskip=3mm,
	belowskip=3mm,
	showstringspaces=false,
	columns=flexible,
	basicstyle={\small\ttfamily},
	numbers=none,
	numberstyle=\tiny\color{gray},
	keywordstyle=\color{blue},
	commentstyle=\color{dkgreen},
	stringstyle=\color{mauve},
	breaklines=true,
	breakatwhitespace=true,
	tabsize=3
}


\usepackage{graphicx}
%Path in Windows format:
\graphicspath{ {images/} }

\usepackage{subfig}

\usepackage{hyperref}
\hypersetup{
	colorlinks=true,
	linkcolor=blue,
	filecolor=magenta,      
	urlcolor=cyan,
}

\begin{document}
	\maketitle
	\tableofcontents
	
	
	
	\clearpage
	
	\section{Overview}
	
	Youtube Video Link: \href{https://www.youtube.com/watch?v=zhsaB9KtgVE}{https://www.youtube.com/watch?v=zhsaB9KtgVE}\newline
	Github Link: \href{https://github.com/wandyezj/ArduinoFaceTracker}{https://github.com/wandyezj/ArduinoFaceTracker}\newline
	\newline
	Code Overview:
	\newline
	ArduinoHardwareControl - RedBearDuo Arduino Hardware code
	\newline
	FaceTrackerBLE - Android Application Code
	\newline
	\newline
	
	
	
	\section{Creative Feature}
	\section{Challenges}
	
	\begin{itemize}
		\item Figuring out that the RedBearDuo only supported the piezo buzzer on certain pins.
		\item Figuring out that the motor power source needed to share the same ground as the control.
		\item Constructing an draft holder: attaching the motor to the ultrasonic sensor.
		\item Figuring out why the face sensor on Android suddenly stopped working.
	\end{itemize}
	
	
	\section{Reflection}
	
	% (i) provides an overview of your design; 
	%(ii) describes your creative feature; 
	%(iii) enumerates key struggles and challenges; 
	%(v) reflects on what you learned. You should include as many images as you want (at least one) that helps explain your night light. Images are free. You need not write in prose (you can bullet point the entire report). Please use headers to clearly separate the four sections of your report.
\end{document}